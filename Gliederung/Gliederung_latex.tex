\documentclass{article}

\usepackage[ngerman]{babel} %deutsche Texte (Inhaltsverzeichnis)
\usepackage[utf8]{inputenc} %UTF8 benutzen
\usepackage{hyperref} %für links

%fehlt noch betreuer, fach, seminararbeit, institut, ort
\title{Wie wichtig ist die O-Notation}
\author{Tobias Schneider, Fatih Kahraman}
\date{\today}




\pagenumbering{roman}
\begin{document}

\maketitle % setzt title author und date
\thispagestyle{empty} %Diese Seite ohne Seitenzahl
\newpage{}

\tableofcontents{}% 2 mal ausführen für richtige darstellung des Inhaltsverzeichnis!!!!!!!!!!!!!!!!!!!!!!
\setcounter{page}{1} % Seitenzahlzähler zurück auf 1 setzen
\newpage{}
\pagenumbering{arabic}


\section{Einleitung}
\subsection{Abstract}
\subsection{Leser Fangen}
\subsection{Problem und Relevanz}

\section{Hauptteil}
\subsection{Definitionen}
\subsubsection{O-Notation}
\subsubsection{In-Place}
\subsubsection{Stabilität}
\subsubsection{Heap/Stack Größe}
\subsubsection{Testumgebung}
SSD vs HDD , CPU, Compiler, Sprache ...


\subsection{Sortieralgorithmus X}
vorstellungen von Unterschiedlichen SA. min 5 bis (Textlimit erreicht ;D  )
\subsubsection{Vorstellung X}
\subsubsection{Pseudo Code}
\subsubsection{Parameter}
O-Notation, In-Place, Stabilität
\subsubsection{Testfälle}
Worst Case, Average Case, Best Case, nearly sorted, festplattenart, genug Speicher, zuwenig Speicher, unterschiedliche Datentypen - Integer versus klassenobjekte

\section{Schluss}
\subsection{Evaluierung}
\subsubsection{Vergleich der Ergebnisse}
\subsubsection{relevanz der O-Notation}
\subsubsection{Wie wichtig sind weitere Kriterien}

\subsection{Zusammenfassung und Ausblick}
\subsubsection{Fazit}
\subsubsection{Anwendungstipps}



\section{Literaturverzeichnis}
vieles kostenlos mit hda Account  \url {www.ieeexplore.ieeee.org} \\ \\
Analysis and Testing of Sorting Algorithms on a 
Standard Dataset \url {http://ieeexplore.ieee.org/document/7280062/}\\ \\
Rheinwerk OpenBook C von A bis Z \url {http://openbook.rheinwerk-verlag.de/c_von_a_bis_z} \\ \\
Rheinwerk OpenBook Java ist auch eine Insel \url {http://openbook.rheinwerk-verlag.de/javainsel} \\ \\
Wikipedia (Übersicht + keine verlässliche Quelle) \url {https://en.wikipedia.org/wiki/Sorting_algorithm}

\section{Anhang}

\section{Eidesstattliche Erklärung}
\end{document}