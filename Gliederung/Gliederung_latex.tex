\documentclass{article}

\usepackage[ngerman]{babel} %deutsche Texte (Inhaltsverzeichnis)
\usepackage[utf8]{inputenc} %UTF8 benutzen

%fehlt noch betreuer, fach, seminararbeit, institut, ort
\title{Sortieralgorithmen - Wie wichtig ist die O-Notation}
\author{Tobias Schneider, Fatih Kahraman}
\date{\today}



\pagenumbering{roman}

\begin{document}
\maketitle % setzt title author und date

\newpage{}
\tableofcontents{}% 2 mal ausführen für richtige darstellung des Inhaltsverzeichnis!!!!!!!!!!!!!!!!!!!!!!
\newpage{}
\pagenumbering{arabic}


\section{Einleitung}
\subsection{Leser Fangen}
\subsection{Geschichte der SA}

\section{Hauptteil}
\subsection{Definitionen}
\subsubsection{O-Notation}
\subsubsection{In-Place}
\subsubsection{Stabilität}
\subsubsection{Heap/Stack Größe}
\subsubsection{Testumgebung}
\subsubsection{Benutzung}

\subsection{Sortieralgorithmus X}
\subsubsection{Vorstellung X}
\subsubsection{Pseudo Code}
\subsubsection{In-Place}
\subsubsection{Stabilität}
\subsubsection{O-Notation}
\subsubsection{Tests} % Genug speicher, zuwenig speicher, bereit sortiert, worst case, festplattenart


\section{Schluss}
\subsection{Evaluierung}
\subsubsection{Vergleich der Ergebnisse}
\subsubsection{relevanz der O-Notation}
\subsubsection{Wie wichtig sind In-Place und Stabilität}

\subsection{Zusammenfassung und Ausblick}
\subsubsection{Mögliche neue Sortieralgorithmen}

\section{Literaturverzeichnis}

\section{Eidesstattliche Erklärung}
\end{document}